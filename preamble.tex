\makeatletter

% usepackage {{{
% 数式ベクトル(太字)
\usepackage{bm}
% AMS math
\usepackage{amsmath,amsfonts,amssymb}
% 擬似コード
\usepackage{algorithm,algpseudocode}
% URL タイプセット
\usepackage{url}
% floating[H]
\usepackage{here}
% 枠環境(framed)
\usepackage{framed}
% TikZ
\usepackage{tikz}
% TikZによる数式装飾
\usepackage[beamer]{hf-tikz}
% ソースコード貼付け
\usepackage{listings}
% 取り消し線
\usepackage{ulem}
% }}}

% Appendixページの番号修正 ref. https://tex.stackexchange.com/a/2559 {{{
\newcommand{\backupbegin}{
    \newcounter{framenumberappendix}
    \setcounter{framenumberappendix}{\value{framenumber}}
}
\newcommand{\backupend}{
    \addtocounter{framenumberappendix}{-\value{framenumber}}
    \addtocounter{framenumber}{\value{framenumberappendix}}
}
% }}}

% beamer template {{{
% テーマ読み込み
\useoutertheme{split}
\useinnertheme{rectangles}
\usecolortheme{orchid}
\usecolortheme{dolphin}
\usefonttheme{professionalfonts}

% タイトルバー背景無効化
\setbeamercolor{frametitle}{bg=}

% ナビゲーションアイコン無効化
\setbeamertemplate{navigation symbols}{}

% caption に番号追加
\setbeamertemplate{caption}[numbered]

% block環境を丸枠,影付に
\setbeamertemplate{blocks}[shadow=true]
\setbeamertemplate{blocks}[rounded]

\setbeamercolor{normal text}{bg=,fg=black}
\setbeamercolor{headline}{ bg=,fg=}
\setbeamercolor{footline}{ bg=,fg=}
\setbeamercolor{titlelike}{bg=,fg=}

% split theme の headlineを改造
\setbeamertemplate{headline}{}

% split theme の footlineを改造
\setbeamertemplate{footline}
{%
    \leavevmode%
    \hbox{%
        \begin{beamercolorbox}[wd=.9\paperwidth,ht=3.5ex,sep=1ex]{footline}%
            \insertshortauthor{} - \textit{\insertshorttitle} (\insertshortdate{})%
        \end{beamercolorbox}%
        \begin{beamercolorbox}[wd=.1\paperwidth,ht=3.5ex,sep=1ex,right]{footline}%
            \insertframenumber{} / \inserttotalframenumber%
        \end{beamercolorbox}%
    }%
}

% }}}

% フォント関連 {{{
\renewcommand{\rmdefault}{ptm} % Roman
\renewcommand{\sfdefault}{phv} % Sans-serif
\renewcommand{\ttdefault}{pcr} % Typewriter
\usepackage{nimbusmono}
\usepackage[T1]{fontenc}
% default
\renewcommand{\familydefault}{\sfdefault}
% }}}

% 日本語関連 (英語の場合はコメントアウト) {{{
% しおりの文字化け防止用
\ifnum 42146=\euc"A4A2 \AtBeginDvi{\special{pdf:tounicode EUC-UCS2}}\else
\AtBeginDvi{\special{pdf:tounicode 90ms-RKSJ-UCS2}}\fi
% フォントマルチウェイト化
\usepackage[deluxe]{otf}
% 日本語デフォルトフォント
\renewcommand{\kanjifamilydefault}{\gtdefault}
% }}}

% 擬似コード {{{
\renewcommand{\algorithmicrequire}{\textbf{Input:}}
\renewcommand{\algorithmicensure}{\textbf{Output:}}
% }}}

% caption {{{
\renewcommand{\figurename}{Fig.}
\renewcommand{\tablename}{Tab.}
% }}}

% 色定義 {{{
\definecolor{darkgreen}{RGB}{0,128,0}
\definecolor{darkblue}{RGB}{0,0,160}
\definecolor{darkred}{RGB}{128,0,0}
\definecolor{darkyellow}{RGB}{192,192,0}
% }}}

% よく使うものを定義 {{{
\newcommand{\strong}[1]{{\bfseries #1}}
\newcommand{\emm}[1]{{\itshape #1}}
\newcommand{\comm}[1]{{\ttfamily #1}}
\newcommand{\commr}[1]{{\ttfamily\color{red} #1}}
\newcommand{\keyword}[1]{{\ttfamily\bfseries\color{blue} #1}}
\newcommand{\func}[1]{{\ttfamily\bfseries #1}}
\newcommand{\insertURL}[1]{ {\scriptsize \url{#1}} }
\newcommand{\eqn}[1]{(\ref{#1})}
% }}}

% 数学記号 {{{
\newcommand{\RR}{\mathbb{R}}
\newcommand{\NN}{\mathbb{N}}
\newcommand{\CC}{\mathbb{C}}
\newcommand{\HH}{\mathrm{H}}
\newcommand{\FF}{\mathrm{F}}
\newcommand{\ii}{\mathrm{i}}
\newcommand{\dd}{\mathrm{d}}
\newcommand{\INV}{{-1}}
% }}}

% sectionの初めにTOCを挿入 {{{
\AtBeginSection[]
{
    \begin{frame}
        \frametitle{Outline}

        \tableofcontents[currentsection]
    \end{frame}
}
% }}}

% Listings {{{
\lstset{%
    basicstyle=\ttfamily\scriptsize,%
    keywordstyle=\bfseries,%
    frame=single,%
    tabsize=4,%
    breaklines=true,%
}
% }}}

\input{info.tex}

\makeatother
